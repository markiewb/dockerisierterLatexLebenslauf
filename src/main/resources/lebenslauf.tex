\documentclass[a4paper,10pt]{article}
\usepackage[bottom=2cm]{geometry}

%A Few Useful Packages
\usepackage{marvosym}
\usepackage{fontspec} 					    %for loading fonts
\usepackage{xunicode,xltxtra,url,parskip} 	%other packages for formatting
\RequirePackage{color,graphicx}
\usepackage[usenames,dvipsnames]{xcolor}
\usepackage[big]{layaureo} 				    %better formatting of the A4 page
% an alternative to Layaureo can be ** \usepackage{fullpage} **
\usepackage{supertabular} 				    %for Grades
\usepackage{titlesec}					    %custom \section

\usepackage{array}
\usepackage{framed}

%Setup hyperref package, and colours for links
\usepackage{hyperref}
\definecolor{linkcolour}{rgb}{0,0.2,0.6}
\hypersetup{colorlinks,breaklinks,urlcolor=linkcolour, linkcolor=linkcolour}

%FONTS
\defaultfontfeatures{Mapping=tex-text}
%\setmainfont[SmallCapsFont = Fontin SmallCaps]{Fontin}
%%% modified for Karol Kozioł for ShareLaTeX use
\setmainfont[
SmallCapsFont = Fontin-SmallCaps.otf,
BoldFont = Fontin-Bold.otf,
ItalicFont = Fontin-Italic.otf
]
{Fontin.otf}
%%%

%deutsche Silbentrennung
\usepackage[ngerman]{babel}
%deutsche Anfuhrungszeichen - https://de.wikibooks.org/wiki/LaTeX-Wörterbuch:_Anführungszeichen
\usepackage[autostyle=true,german=quotes]{csquotes}

%CV Sections inspired by: 
\titleformat{\section}{\Large\scshape\raggedright}{}{0em}{}[\titlerule]
\titlespacing{\section}{0pt}{3pt}{3pt}


%-------------WATERMARK TEST [**not part of a CV**]---------------
\usepackage[absolute]{textpos}

\setlength{\TPHorizModule}{30mm}
\setlength{\TPVertModule}{\TPHorizModule}
\textblockorigin{2mm}{0.65\paperheight}
\setlength{\parindent}{0pt}



%--------------------BEGIN DOCUMENT----------------------
\begin{document}
 %https://de.wikibooks.org/wiki/LaTeX-Wörterbuch:_tabular
 %Zur Ausrichtung von p-Spalten in tabular-Umgebungen
 \newcommand{\ltab}{\raggedright\arraybackslash} % Tabellenabschnitt linksbündig
 \newcommand{\ctab}{\centering\arraybackslash} % Tabellenabschnitt zentriert
 \newcommand{\rtab}{\raggedleft\arraybackslash} % Tabellenabschnitt rechtsbündig
 \newcolumntype{R}[1]{>{\rtab}m{#1}}
 \renewcommand{\footnotesize}{\fontsize{10pt}{11pt}\selectfont}
 
\pagestyle{empty} % non-numbered pages

\font\fb=''[cmr10]'' %for use with \LaTeX command

%--------------------TITLE-------------
 \section{\huge Bewerbung Software Engineer}

 

 % -------------------------------------------------
 % Passfoto
 % -------------------------------------------------

\vskip 8em
\begin{center}
 %\fbox{\includegraphics[width=5cm]{Bewerbungsfoto.png}}
 \includegraphics[width=6.5cm]{Bewerbungsfoto-sw.png}
\end{center}


 \vskip 3em

\par{\centering {\huge Maik \textsc{Mustermann} }\bigskip\par}

 \vskip 6em

 \section{}

\vskip 2em

 \begin{tabular}{R{3.3cm}l}
  \textsc{Adresse:}          & Dorfstr. 14, 09234 Musterstadt \\
  \textsc{Kontaktdaten:}     & \Telefon~0176 / 12 32 43 44\\
  \textsc{}                  & \Letter~\href{mailto:maik.mustermann@emailprovider.de}{maik.mustermann@emailprovider.de}\\

  \\
  \textsc{Anlagen:}         & Lebenslauf, Referenzen \\
 \end{tabular}
 \section{}


 \newpage
 
%--------------------SECTIONS-----------------------------------
%Section: Personal Data
\section{\huge Lebenslauf}

%Section: Work Experience at the top
\section{Berufserfahrung}
\begin{tabular}{R{1.5cm}|p{12.0cm}}
 \emph{Aktuell} & \textsc{Java-Entwickler} (Senior) bei \textsc{Blablabla GmbH}, Bielefeld \\\textsc{Apr 2066}&
 \\&
 \footnotesize{

 Weiterentwicklung des Produktes \enquote{XXX Platform} (\href{https://www.xxx-platform.com}{Link}), einem System fuer blablaba..
 😀
 \vskip 1em
 Die Hauptaufgabe ist die Umsetzung von ... in einem ... Team.
 \vskip 1em
 Weitere Aufgaben, ...
 \vskip 1em

 (Spring-Hibernate-Tomcat-Stack; Messaging: xxxx, Webservices; UI: xxx, yyy, zzz; Testing: JUnit, xxx, xxx)
}\\\multicolumn{2}{c}{} \\

 
 \textsc{Nov 2055} & \textsc{Build-Ingenieur} bei \textsc{Joes Bude GmbH}, Duesseldorf\\&\emph{ }\\&\footnotesize{Migration der Build-Pipeline und ...
 }\\\multicolumn{2}{c}{} \\
 
 \textsc{Nov 2044} & \textsc{Java-Entwickler} bei \textsc{PizzaDienst AG}, Koeln \\&\emph{ }\\&\footnotesize{

 Neuentwicklung von \enquote{Pizza-Platform} ..., einem Werkzeug zur .... \vskip 1em
 Als projektverantwortlicher Entwickler waren, ... weitere Aufgaben.
 \vskip 1em


 (Spring-Hibernate-Tomcat-Stack; Messaging: xxxx, Webservices; UI: xxx, yyy, zzz; Testing: JUnit, xxx, xxx)
 }\\\multicolumn{2}{c}{} \\
 

\end{tabular}

\section{Praktika und Nebentätigkeiten}
\begin{tabular}{R{1.5cm}|p{12.0cm}}

 \textsc{2022-2022} &  \textsc{Diplomand} bei \textsc{XXX Systemhaus GmbH}, Koeln\\&\emph{ }\\&\footnotesize{Masterarbeit sowie Prototypentwicklung zur 
 \vskip 1 em
 (Java SE, ..., ...)}\\\multicolumn{2}{c}{} \\

 \textsc{2011-2011} & \textsc{YYY-Entwickler} bei \textsc{YYY Center GmbH}, Bonn\\&\emph{ }\\&\footnotesize{
 Nebentätigkeit und Praktikumssemester: Weiterentwicklung des Produktes \enquote{\mbox{SuperTool 2000}} (\href{http://www.supertool2000.com}{Link}), einem ... zur ... für ....  
  \vskip 1 em
  (C\#, .., .. )
 }\\\multicolumn{2}{c}{} \\

 \textsc{2000-2000} & \textsc{YYY-Entwickler} bei \textsc{ZZZ GmbH}, Darmstadt \\&\emph{ }\\&\footnotesize{Nebentätigkeit: Aufbau der Plattform \url{http://www.meine-platform.com} zur ....
 \vskip 1 em
 (...-Stack, JS, .., ..)
 }\\\multicolumn{2}{c}{} \\

\end{tabular}


%Section: Education

\section{Akademische Laufbahn}
\begin{tabular}{R{1.5cm}|p{12.0cm}}

 \textsc{2088} &  XXX XXX-Konferenz 2088\\&\emph{ }\\&\footnotesize{Artikel: \enquote{Blabla fuer Blabla unter dem Kontext von Blabla im Blabla}}\\\multicolumn{2}{c}{} \\

 \textsc{2077} & \textsc{Informatik Master} (\enquote{mit Auszeichnung}), \textsc{Hochschule Musterstadt}\\&\emph{ }\\&\footnotesize{
 Masterarbeit: \enquote{Optimierung von Blabla fuer Blabla in Blabla unter Beachtung von Blabla-IT-Kram}\vskip 0.5 em (Partner \textsc{XXX Systemhaus GmbH})
 }\\\multicolumn{2}{c}{} \\

 \textsc{2066} & \textsc{Informatik Bachelor} (\enquote{mit Auszeichnung}), \textsc{Hochschule Musterstadt}\\&\emph{ }\\&\footnotesize{Bachelorarbeit: \enquote{Praktischer Einsatz von Blamuster- Ansätze zur Lösung von Bla-Problemen im Blabla-Kontext}\vskip 0.5 em (Partner \textsc{YYY Computer GmbH})
 }\\\multicolumn{2}{c}{} \\

\end{tabular}


%Section: Languages
\section{Fremdsprachen}
\begin{tabular}{rl}
\textsc{English:}&Flüssig Niveau XX\\
\textsc{Russisch:}&Niveau XX\\
\end{tabular}

\section{Interessen und Aktivitäten}
 Zur Weiterbildung ... engagiere ich mich in der Weiterentwicklung von Open-Source-Software (\href{https://github.com/meinprofil}{Github-Profil}), seit 20xx besonders für XXXX.
 So bin ich xxxx und ich \href{https://meinblog.wordpress.com}{blogge} darüber.
 Durch ....
 So habe ich mehr Zeit für meine anderen Interessen wie ... und der Suche nach ...

 
\end{document}
